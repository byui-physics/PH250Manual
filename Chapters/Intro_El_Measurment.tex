%Intro_El_Measurment
\chapter{Electrical Measuring Devices}

\objectives
{
\item Demonstrate the use of a DC power supply and a signal generator.
\item Use a multimeter to measure DC voltage and current.
\item Demonstrate the basic functionality of an oscilloscope, including
channel selection, setting the volts per division, the time per division, and
the triggering level.
\item Use an oscilloscope to measure DC voltage and AC frequency and voltage.
\item Demonstrate how current measurements are derived from voltage 
measurements by building a simple ammeter device, utilizing a stand alone
volt meter.
\item In each of these activities, identify and quantify the sources of
uncertainty in your measurements.
}

\review
{
\item Electric potential and potential difference.
\item Electric current.
\item Resistance and Ohm's law.
\item AC and DC potentials.
}

Last week we tackled very simple computer control, but we said we also want
to transfer data from our experiment to our computer. In order to understand
how to do that, we need to know what things we can measure with electronic
devices. We will take on this question today, but we will get practice
making these measurements with special equipment designed just for making
these measurements. These devices won't send the data they measure to our
computers, but they will display it so we can see the data.

Once we know how to make some basic measurements with these stand-alone
instruments, then we can consider how we would make a new instrument to
measure something else. We will build a current measuring device, an \emph{%
ammeter} out of electrical components and a \emph{voltmeter}. This will be
something we do over and over again - building new instruments using
instruments we already know and some electrical equipment.

This lab consists of a large pre-reading section that will give you
background information. It will describe the various stand-alone devices, along
with some electromagnetic theory. Scattered throughout, you will be asked to 
practice making these measurements with our equipment. 

\section{What We Measure: Voltage
\label{Voltage Measurement
with Meter}}

Ultimately, all of our electronic sensors measure one thing: voltage. 
You may want to measure something else, say relative humidity.
To record relative humidity on our computer we need to first
(somehow) convert relative humidity into a voltage. 

Voltage is really \textquotedblleft electrical potential
difference,\textquotedblright\ which is the difference between electrical
potential energy per unit charge at two different circuit locations. Last
lab we said voltage was a comparison, and this is the comparison. We compare
the potential energy at two different circuit locations, only we divide the
potential energy by the charge of an electron.

To get a feel for how this works, think of a change in gravitational
potential energy, $\Delta U_{g}$. If we wanted to measure the difference in
potential energy between the top of a hill and the bottom of a hill, we
would need to place some sort of device both at the top and at the bottom of
the hill, as in Figure \ref{fig:grav_energy}.
\begin{figure}[htbp!]
\centering
\includegraphics[width=0.8\textwidth]{PH4CAU0F}
\caption[Gravitational energy difference]{Gravitational energy
difference. In order to measure the difference in gravitational potential 
energy from the bottom of the hill to the top, we would need to know the
altitude at each position. Note that the difference in the two potential
energies is independent of where zero point of the $y$ coordinate is.}
\label{fig:grav_energy}
\end{figure}

We have to do the same thing in
our electrical case. We need two \textquotedblleft
probes,\textquotedblright\ one placed at the high potential and one placed
at the low potential. For example, we could have the circuit that you see in
Figure \ref{fig:basic_circuit_voltmeter}.
\begin{figure}[hbp!]
\centering
\includegraphics[width=0.8\textwidth]{PH4CAU0G}
\caption[Electric potential difference]{Electric potential difference.
As with gravitational energy, we measure the potential at two different points
and compare the two measurements.}
\label{fig:basic_circuit_voltmeter}
\end{figure}
The positive end of the battery
is like the top of the hill. It provides a high electrical potential energy.
So we put one probe at the top of the \textquotedblleft
hill\textquotedblright\ or the plus side of the battery, and the other on
the bottom of the \textquotedblleft hill\textquotedblright\ or minus side of
the battery. The negative side of the battery provides a low electrical
potential energy. \emph{\ }With this we measure how high our potential
\textquotedblleft hill\textquotedblright\ is. The difference between these
two measurements is called \emph{voltage. } 

You should ask yourself
\textquotedblleft what would happen if you got the probes
backward?\textquotedblright\ The potential difference is the same, but the
probe we thought was at the high potential is at the low potential, and
\textit{vice-versa}. The result is that our reading will change by a 
negative sign.

In Figure \ref{fig:measuring_voltage} 
you can see how to actually perform this voltage
measurement with one of our meters.
\begin{figure}[htbp!]
\includegraphics[width=\textwidth]{PH4CAU0H}
\caption[Measuring voltage with a multimeter]{Measuring voltage with a 
multimeter. The red probe is placed on the lead of the resistor connected
to the high potential of the power source, and the black probe is placed on
the lead connected to the low potential of the power source.}
\label{fig:measuring_voltage}
\end{figure}

We say we measure voltage \textquotedblleft across\textquotedblright\ a
circuit element. This makes some sense if you consider that we very seldom
stand batteries up so their electric potential is greater in the same
direction as their gravitational potential. Batteries, resisters,
capacitors, etc., often lie down, and we measure \textquotedblleft
across\textquotedblright\ them by putting the positive probe on the high
potential side and the negative probe on the low potential side. Even though
the battery is lying down, we are still measuring a higher and lower
potential energy difference. Knowing a little about voltage, let's now look at
our devices that produce voltages and then the devices that measure voltages.

\section{Sources of DC and AC potentials}

In today's lab we will study four hardware devices. A power supply, a signal
generator, a multimeter,and an oscilloscope. We will call these
\textquotedblleft stand alone\textquotedblright\ instruments because they
are independent boxes that do their job of measuring or generating signals
without a computer connected to them. The power supply and the signal
generator make voltage signals. The other two devices measure them. Let's
look at the power supply and signal generator first, then take on the
measuring devices.

\subsection{DC Power Supply}

When most people think about sources of DC potentials, they will first 
think about batteries. Of course, we use batteries in many applications in
today's world, and they power circuits designed to work on a single, specific
DC voltage. 

A DC power supply is like an adjustable battery.
An example can be seen in Figure \ref{fig:dc_power_supply}.
Usually a power supply takes
electrical energy from a wall outlet and converts that energy into the
specific voltage range that we want for our experiment. So it is like a
battery, but must be plugged into the wall. Our power supplies are designed
to keep us safe. They are current limited, meaning that they try not to give
too much charge flowing through our wires. (Sometimes this is a problem
because they are too limited.) There is a current limiting knob that you can
turn to allow a little more current. Be careful when you use this. The
voltage may jump wildly when you turn the current knob! It is best to turn
all the knobs down as low as they will go before you turn on the power
supply. Then, after turning on the power supply, increase the current knob
about half a turn and then slowly turn the voltage knob up to your desired
voltage. If the voltage stops increasing, turn the voltage knob back down a
bit, and turn up your current limiter knob some more. Then try your voltage
knob again. Other devices are typically connected to a power supply using
wires with ``banana'' connectors at the end.
\begin{figure}[htbp!]
\centering
\includegraphics[width=0.8\textwidth]{PH4CAU0I}
\caption[A DC power supply]{A DC power supply.}
\label{fig:dc_power_supply}
\end{figure}

Some of our electrical devices are quite delicate, and will literally burn
up if you apply too much current or voltage. In today's lab, we will
practice using our power supply so we are prepared when the delicate
components come out later.

\subsection{Signal Generator}

The signal generator is basically a fancy
power supply. 
An example can be seen in Figure \ref{fig:signal_generator}.
It makes voltages that change over time in sine,
square, and triangle patterns. These time-varying signals have a maximum
voltage (called the amplitude). We will use both the wave output and a
timing signal that the wave generator creates. Each has their own Bayonet
Neill--Concelman connector (usually just called a BNC connector) on the
front of the signal generator. You will need a cable with BNC connectors on
one end (and maybe alligator clips on the other end) to use this device.
There is an amplitude knob on the front of the signal generator. Because the
signal generator makes a voltage that changes in time, the amplitude of the
signal must be in voltage units. We should be careful not to set the signal
amplitude (voltage) too high or we run the risk of destroying our measuring
devices. Again turn the amplitude (voltage) down before you connect the box
to our electrical components. Then turn up the voltage to what you want in a
safe way.
\begin{figure}[htbp!]
\centering
\includegraphics[width=0.8\textwidth]{PH4CAU0J}
\caption[A signal generator]{A signal generator.}
\label{fig:signal_generator}
\end{figure}

Different brands and types of signal generators are operated differently.
We will focus on the one shown in Figure \ref{fig:signal_generator}.
There are frequency range buttons (using the shift button) near the middle
of the device panel. To change the frequency, you use the shift and range
buttons to set which digit you are adjusting, then turn the frequency knob
to make the change. An annoying feature of our frequency generators is that
you must push the \textquotedblleft output on\textquotedblright\ button or
they don't output a signal. When everything is set up right, we get a sine
wave (or square wave, or triangle wave) out. 
Figure \ref{fig:signal_generator_output} shows a signal from the
signal generator displayed on one of our measuring devices, the
oscilloscope. 
\begin{figure}[htbp!]
\centering
\includegraphics[width=0.8\textwidth]{PH4CAU0K}
\caption[Sine wave output from a signal generator]{Sine wave output from
a signal generator, displayed on an oscilloscope.}
\label{fig:signal_generator_output}
\end{figure}

A household power outlet is also a source of sinusoidally varying voltage, but
with two limitations. First, the amplitude of the signal is (hopefully) always
at the same set value (170 Volts in the United States), and the sine wave
always has the same frequency (60 Hz in the United States). A signal generator 
provides far more flexibility in both the amplitude and frequency of the signal,
as well as the wave form.

\section{Instruments for Measuring Voltage}

Now that we've met a few sources of voltages, lets consider some stand alone
devices that can measure those voltages.

\subsection{Voltmeters}

Our first measurement device is voltmeter. It measures the electric
potential (voltage) between its two leads (sometimes called
\textquotedblleft probes\textquotedblright ). Figure
\ref{fig:voltemeter_one} shows an example of a
multimeter set to measure voltage.
The display is set to read
voltage by turning the dial to the $\unit{V}$ position. There are often two
voltage settings. The one that has a wavy line next to it is alternating
current (AC) voltage. 
The one that has a straight line with three dots under it is the
direct current (DC) voltage. 
For now, we will just use the DC
voltage setting. 
The leads (probes) should be connected to the COM (common) and $%
\unit{V}\unit{%
%TCIMACRO{\U{3a9}}%
%BeginExpansion
\Omega%
%EndExpansion
}\unit{Hz}$ connectors. 

\begin{figure}[htbp!]
\centering
\includegraphics[width=0.8\textwidth]{PH4CAU0L}
\caption[A multimeter set to measure DC voltage]{A multimeter set to 
measure DC voltage.}
\label{fig:voltemeter_one}
\end{figure}

Note that connecting your probe leads to the wrong
position can blow the fuse (or worse) in your meter, causing the meter to
stop working altogether or display incorrect measurements. 
You should make sure you don't do this, and watch to
make sure someone else has not done this before you. If the meter seems to
be behaving oddly, it may be an indication someone who used it before you
connected the probes incorrectly! 

For reference, Figure \ref{fig:voltmeter_two} shows a
picture of a different model of multimeter.

\begin{figure}[htbp!]
\centering
\includegraphics[width=0.8\textwidth]{PH4CAU0M}
\caption[A different model of multimeter]{A different model of multimeter,
also set to measure DC voltage.}
\label{fig:voltmeter_two}
\end{figure}

You will notice that multimeters have
other settings besides volts. They are designed to be able to measure more 
than one thing. Ultimately, however, each of these other measurements are
in actuality derived from voltage measurements. We will likely use some of
the other settings on the multimeter during the semester.

\subsection{Oscilloscopes}

Our next device, the oscilloscope, is just a fancy voltmeter. A voltmeter
only displays a single number, giving us no additional details other
that what the currently measured voltage is. The oscilloscope measures 
voltage \textit{as a function of time}, providing us with significantly more
information. The downside is that, where a multimeter is pretty simple to 
set up and use, an oscilloscope is a far more complex peice of equipment.

The oscilloscope can accurately and precisely measure changing voltages 
and usually
has a way to graph the changing voltage as a function of time. 
A sinusoidally varying voltage should look something like Figure 
\ref{fig:sine_voltage} when
plotted. 
This plot represents what our oscilloscope
does. We should see something like Figure \ref{fig:sine_voltage} 
on the oscilloscope screen. 
In figure \ref{fig:signal_generator_output}, found in our
discussion of the signal generator, you know that this is just what we see.
\begin{figure}[htbp!]
\centering
\includegraphics[width=0.8\textwidth]{PH4CAU0N}
\caption[Sinusoidally varying voltage]{Sinusoidally varying voltage. This
is the type of information an oscilloscope is designed to display.}
\label{fig:sine_voltage}
\end{figure}

If the changing voltage is periodic, the oscilloscope has a way to use this
fact to stabilize the graph so you can see the details more clearly. This
stabilization is called \textquotedblleft triggering\textquotedblright\ and
on our oscilloscopes there are buttons and knobs on the right hand side of
the oscilloscope that adjust the triggering to make the graph more stable
(or less stable). The photograph of the sine wave 
found in Figure \ref{signal_generator_output} was taken by
stabilizing a sine wave from a signal generator. The oscilloscope starts
plotting at the same part of the wave each time, so the periodic signal
seems to stand still. To do this we must \textquotedblleft
trigger\textquotedblright\ the graph at some good starting point. Our
oscilloscopes have a built-in circuit that can watch for the same part of a
signal and start the graph in the same place each time. One of the knobs
adjusts the trigger point. This knob can be seen in Figure
\ref{fig:oscilloscope_control_a}.
\begin{figure}[htbp!]
\centering
\includegraphics[width=\textwidth]{PH4CAU0P}
\caption[Some of the controls on an oscilloscope]{Some of the controls
on an oscilloscope, including the scale adjustments and triggering controls.}
\label{fig:oscilloscope_control_a}
\end{figure}

The other labeled controls in Figure \ref{fig:oscilloscope_control_a} adjust the
horizontal and vertical axes. The vertical axis is voltage, and the voltage
axis control is next to the signal input toward the bottom middle of the
front panel. To the right of this is the horizontal axis control, which is
time. You can choose how many volts per division with one knob and how many
seconds (or fractions of seconds) per division you have on your graph with
another knob. 

% FIXME Is this information correct? I'm pretty sure the oscilloscope 
% display is for the volts/division.
In addition to displaying the measured voltage signal, the oscilloscope screen
shows us information about the settings of the oscilloscope. In Figure
\ref{fig:oscilloscope_display_a}, at the lower left hand corner, 
there is a red dot next to a number. This number tells us the voltage
represented by each vertical division (square) on the display. 
When the picture in Figure \ref{fig:oscilloscope_display_a} was taken,
the voltage knob was set to 2 V per division, so the amplitude of this 
signal is somewhere around 3 V.
\begin{figure}[htbp!]
\centering
\includegraphics[width=\textwidth]{PH4CAU0Q_corrected}
\caption[The display of an oscilloscope]{The display of an oscilloscope, with 
the volts per division set to 2 V. The signal seen on the generator has an
amplitude of approximately 1.5 divisions, or 3 V.}
\label{fig:oscilloscope_display_a}
\end{figure}

%\begin{figure}[h!]
%\includegraphics[width=3.9364in,height=%
%3.3076in]{PH4CAU0R}
%\end{figure}

Most oscilloscopes have two signal inputs, meaning we can look at two
different voltage signals at the same time. Each signal input is called a
\textquotedblleft channel.\textquotedblright\ Each channel has it's own
voltage scale knob and voltage scale indicator in the bottom left-hand
corner. They share the same time scale.
The channel inputs each have a BNC connector. We use oscilloscope probes
connected to these connectors. 
If we want to measure two voltages at once, we need
two sets of probes!

To check that our oscilloscope is working correctly we can measure a known
voltage, say, the voltage of a regular battery for example. What would we 
expect to see? Since the battery provides a constant voltage, we should
simply see a straight line on the oscilloscope, with the position of the line
corresponding to the voltage of the battery. For example, if I was using a 
1.5 V battery, and had the oscilloscope set to 0.5 Volts per division, I
would expect to see a straight line at the third division on the display.

%\begin{figure}[h!]
%\includegraphics[width=4.0836in,height=1.9458in]{PH4CAU0S}
%\end{figure}Notice that I changed the voltage
%scale knob position so that now the oscilloscope screen has a $10\unit{V}$
%total potential change. That means that we have $5\unit{V}$ at the top of
%the screen and $-5\unit{V}$ at the bottom of the screen. The screen is
%divided into little boxes. There are five rows of boxes from the bottom to
%the top of the screen. Each box represents $1/10$ of the total voltage.
%Since we have $\Delta V=10\unit{V},$ each box represents $\Delta V=1\unit{V}%
%. $ So our battery voltage should give us one and a half boxes. And that is
%just what we got.

Sometimes the oscilloscope does not get the right voltage. If this
happens we need to calibrate the oscilloscope. (Every time we use an
oscilloscope it is a good idea to check it to make sure it working well.) Our
oscilloscopes have a test voltage to use just for this purpose. The location
of the calibration signal source can be seen in Figure
\ref{fig:oscilloscope_calibration}. 
The calibration source makes a $5\unit{V}$ square wave. 
By connecting the channel probes to the calibration source, a square wave
should be seen on the display (you may have to change the time per division
to see the wave form). If the top of the wave crosses at 5V on the display,
and the bottom of the wave crosses at the voltage axis, then the oscilloscope
is properly calibrated. If the oscilloscope is not properly calibrated, you
can find the procedures for doing so in the user manual (search 
online for the manual
of the model you are using).
\begin{figure}[htbp!]
\centering
\includegraphics[width=\textwidth]{PH4CAU0T}
\caption[Oscilloscope calibration source]{Oscilloscope calibration source, 
built in to the oscilloscope. The source produces a 5V square wave, which can
be tested using the channel probes.}
\label{fig:oscilloscope_calibration}
\end{figure}

%(try it to see what
%that looks like!). If we use this calibration source we should get something
%like what you see in the next figure. \begin{figure}[h!]
%\includegraphics[width=4.2004in,height=1.9623in]{PH4CAU0U}
%\end{figure}If you don't get $5\unit{V},$
%then some thing is wrong and you will need to go through the oscilloscope's
%calibration procedure. That is in the oscilloscope manual and you can find
%the manual on-line.

\section{Measurement Uncertainty: A Review}

As you are aware, every measurement comes with some uncertainty, and voltage
measurements are no exception. The multimeters only display voltages to a 
specific decimal point, leaving us uncertain of what the next digit would be.
Thus, the precision of the last digit of the display is the minimum amount
of uncertainty we would have in that measurement. 

There are potentially other sources of uncertainty as well. If I was using an
analog voltmeter (something we won't do in this lab), I would be making a 
judgment about which mark on the scale the needle was the closest to. This is
akin to using a ruler to measure a distance. 

Digital multimeters, such as ours, operate using an analog-to-digital converter
(ADC). We'll be learning more about ADCs later in the semester. For now, it is
sufficient to note that an ADC introduces some additional uncertainty into 
measurements obtained using digital voltmeters. (However, the relative size 
of these ADC uncertainties in a modern digital voltmeter is probably far less
than the uncertainty from the limited number of digits on the display).

Note that at this point we haven't even considered the question of whether
the meters are reading accurately or not. It is always a good idea to check
the calibration of your equipment!
 
The oscilloscope presents some additional challenges when it comes to 
uncertainty. It doesn't provide a nice digital readout like the multimeter.
When we ``read'' voltages or times from the oscilloscope, we are making
a judgment as to which division the signal is the closest to. Even if I'm 
pretty good at making judgments, I probably can't depend on my measurement 
being any more precise than the smallest division mark on the screen.

Keep these uncertainties in mind as you complete the following activities.

\activity{
Practice measuring voltage using a multimeter and an oscilloscope.
\begin{itemize}
\item Measure the voltage of a D cell or 9V battery using the multimeter. How
certain is this measurement? What measurement do you get if you reverse the
probes of the multimeter? (Remember that by ``reversing'' the probes, it means
you touch the ends of the probes to the ``wrong'' sides of the battery. Do 
not reverse the ports that the probes are plugged into on the multimeter!)
\item Build the circuit from Figure \ref{fig:measuring_voltage}, and provide
a modest amount of voltage to the resistor using the power supply. (If you
put too much current through a resistor, it will literally burn up. Keep the
current down to a few hundred milliamps at most.) Use the multimeter to measure
the voltage across the resistor. Compare this reading to the voltage reading 
on the power supply. Do they agree? Do they agree within the uncertainties? If
they did not agree, which reading would you trust more? Why?
\item Measure the voltage of a battery using an oscilloscope. How 
certain is this measurement? Can you improve the relative precision of the
measurement by changing the settings on the oscilloscope? What happens to the
measurement if you reverse the probes? Which device is typically better for
measuring battery voltage: the multimeter or the oscilloscope? Why?
\item Generate a sinusoidal signal using a signal generator. Display the signal
on the oscilloscope. Report the amplitude and frequency of the signal as 
measured by the oscilloscope, along with their uncertainties. Compare these
readings to those indicated by the signal generator.
\end{itemize}
}



% FIXME Start here!


\section{Measuring Something Else: Current}

Our multimeters have a current setting as well as a voltage setting. Current
is a flow of charge. This is like a water current, which is a flow of water,
only we have charge flowing instead of water. We can write the flow of charge
as
\begin{equation}
I=\frac{\Delta Q}{\Delta t}
\end{equation}
where for us $\Delta Q$ is the amount of charge that has gone by in the time 
$\Delta t.$ Physicists normally use the letter $I$ to represent electrical 
current.

We should take a minute to think about what to expect when we allow charge
to flow. Think of a garden hose. If the hose is full of water, then when we
open the faucet, water immediately comes out. The water that leaves the
faucet is far from the open end of the hose, though. We have to wait for it
to travel the entire length of the hose. But we get water out of the hose
immediately! Why? 
The new water coming in causes a
pressure change that is transmitted almost instantly 
through the hose, and the water at the open
end is pushed out. You can tell this is the case because the water
immediately leaving the hose is warm and tastes like plastic hose. After a
while, the water is colder and cleaner.
%\begin{figure}[h!]
%\includegraphics[width=1.4547in,height=1.8228in]{PH4CAU0V}
%\end{figure}

Current behaves in the same way. When we flip a light switch, the
electrons near the switch start to flow. But there are already free
electrons in the wire. These experience a push that makes the light turn on
almost instantly. But the electrons that turn on the light are not the ones
that just went through the switch.

Pictured another way, turning on the switch connects the wires to an electric
potential. This potential results in an electric field propagating very 
quickly (like at speeds close to the speed of light) through the wires. As 
soon as that electric field hits any point of the wire, the free 
electrons there begin to move.

\subsection{Measuring Current}

Because current is a flow, to measure current we must put a meter into that
flow. In a house, if you want to measure how much water is used, you connect
the pipe from the city water system to a meter. The water flows through the
meter and then goes into the pipe that brings water to the house. That way,
the meter can't miss any of the water (and the city can't miss any of your
payment!). The same is true for electrical current. To measure electrical
current, we need to break open the circuit and install a meter through which
all of the current must flow, as in Figure \ref{fig:measuring_current}.
In this diagram, you can see that
the electric current must go through the current meter. In fact, it couldn't
go anywhere else because part of the original circuit wire is missing. This
is just what we want. 
\begin{figure}[htbp!]
\centering
\includegraphics[width=0.8\textwidth]{PH4CAU0W}
\caption[Measuring current]{Measuring current. The current meter must
be positioned such that all of the current flows through the meter.}
\label{fig:measuring_current}
\end{figure}

To actually perform this measurement with one of our
multimeters you could set up a circuit like the one in Figure
\ref{fig:measuring_current_b}. 
There is one more important thing
to do to make this work. We need to change the meter settings. And there are
two separate changes. The first is to switch the probe connections. One
probe stays in the COM or common connector, but the other needs to move to
the connector marked with an \textquotedblleft A\textquotedblright\. 
The \textquotedblleft
A\textquotedblright\ stands for the standard unit of electrical current, the
Ampere or Amp. With the multimeter set up like this we would call it an 
\emph{ammeter}. Ammeters measure electrical current.
Figure \ref{fig:multimeters_amps} shows the probe connections for measuring
current with two kinds of multimeters. 
\begin{figure}[htbp!]
\centering
\includegraphics[width=0.8\textwidth]{PH4CAU0X}
\caption[Measuring current with a multimeter]{Measuring current with a
multimeter.}
\label{fig:measuring_current_b}
\end{figure}
\begin{figure}[htbp!]
\centering
\includegraphics[width=\textwidth]{PH4CAU0Y}
\caption[Two multimeters prepared for measuring current]{Two multimeters
prepared for measuring current.}
\label{fig:multimeters_amps}
\end{figure}

Many
multimeters have two ports for measuring current, one for milliamps (mA), and
one for Amps (A). If you don't know what the current is that you are trying 
measure, it's always a good idea to start with the probe connected to the port
designed to measure Amps. Too high a current will burn a fuse in the 
multimeter, or worse yet, destroy the multimeter.

\subsection{How an Ammeter Works}

We said before that physicists like to change any measurement they can into
a voltage measurement, because, ultimately, that is the only thing that we
can measure with electronics. So, if we want to measure current, the first
thing we need to do is transform a current into a voltage.

This is true not only for current, but for any other quantity that we would
like to measure electroncially. 
In order to build a new instrument, we need to understand the physics of the 
thing that we really want to measure. 

\subsubsection{Ohm's Law}

Let's keep thinking of current like water in a hose. Will there be any
friction associated with the water traveling through the hose? Of course
there will! We usually call friction in fluids \emph{viscosity}. But it is a
form of friction, and we can use our intuition about friction to see
how it would work. Think of having two hoses, one twice as long as the
other. Which would you expect to have more friction? 
%\begin{figure}[h!]
%\includegraphics[width=2.6033in,height=%
%1.7474in]{PH4CAU0Z}
%\end{figure}
Our friction experience says that the longer the path, the more the
current interacts with the hose. The more the current interacts with the hose,
the more friction it will experience. 
Electrical currents are like this. Longer wires
give more friction.

George Simon Ohm noticed that with long metal wires, there seemed to be a
linear relationship between the potential difference (voltage), the current,
and the length of the wire. The longer the wire, the less current a given
voltage would produce. His
work was confirmed and expanded on by others, who found that not only length
mattered, but also the diameter of the wire mattered. The relationship is
now expressed as
\begin{equation}
\Delta V=IR
\end{equation}
an equation referred to as \emph{Ohm's law}. The
$\Delta V$ is our old friend, voltage, and the $I$ represents
current. The $R$ in the equation represents the ``friction'' provided by the 
ciruit.
Experiments show that this constant $R$ depends on the material. It is like
our viscosity in hoses. It is the friction. The more the friction, the
harder it is to get the current through the wire. But like we don't call
viscosity \textquotedblleft friction,\textquotedblright\ we also don't use
the word \textquotedblleft friction\textquotedblright\ for this
friction-like term. We call it \emph{resistance.} We could solve for this
resistance 
\begin{equation}
R=\frac{\Delta V}{I}
\end{equation}
or we could plot $\Delta V$ vs. $I$ and the slope of this line would be the
resistance.
Either way, this relationship
tells us that it takes more potential energy to get the same current if
there is more resistance.

%\begin{figure}[h!]
%\includegraphics[width=1.9899in,height=1.3214in]{PH4CAU10}
%\end{figure}

Ohm's law holds well for metals and
many materials, but, like Hooke's law, this \textquotedblleft
law\textquotedblright\ does not always hold. Devices that do provide a
constant resistance coefficient $R$ are called \emph{resistors}. In schematic
diagrams, resistors are usually represented by zig-zag lines or by rectangles.
In reality, they look more like Figure \ref{fig:resistor}.
\begin{figure}[htbp!]
\centering
\includegraphics[width=0.6\textwidth]{PH4CAU11}
\caption[A resistor]{A resistor. The colored bands indicate the resistance.}
\label{fig:resistor}
\end{figure}

\subsubsection{Resistor Code}

Many commercially produced resistors come conveniently marked with a color
code that helps you identify their resistance. The basics of the color code
are seen in Figure \ref{fig:resistor_code}.
\begin{figure}[htbp!]
\centering
\includegraphics[width=0.8\textwidth]{PH4CAU12}
\caption[Resistor code]{Resistor code. The use of this code is
described in the text.}
\label{fig:resistor_code}
\end{figure}

To use the code, do the following:
\begin{enumerate}
\item Find the tolerance code band. This band is usually brown in our kit
resistors and often is set off from the others a little more.
\item Read the first color band from the side opposite the tolerance band.
This will be the first digit of your resistance. I\ think the example
resister on the chart has a yellow first color band, so the first digit of
our resistance is $4.$
\item Read the second color band. This will be the second digit of your
resistance. I\ think the second band of our example resistor is orange, so
the second digit would be a $3,$ making our resistance so far $43$
\item Read the third color band. This will be the third digit of your
resistance. I\ think the second band of our example resistor is red, so the
second digit would be a $2,$ making our resistance so far $432$
\item Read the forth color band. This is a multiplier. You multiply the
first three digits by this amount. For our example resistance, I think the
third band is black. Then we multiply $432$ by $1\unit{%
%TCIMACRO{\U{3a9}}%
%BeginExpansion
\Omega%
%EndExpansion
}$ to get $432\unit{%
%TCIMACRO{\U{3a9}}%
%BeginExpansion
\Omega%
%EndExpansion
}.$ This is our resistance.
\item The tolerance band gives the uncertainty in this value. Our example
resistor seems to have a brown tolerance band, which tells us our value is
good to $\pm 1\%.$ For our example resistance, $1\%$ would be $0.01\times 432%
\unit{%
%TCIMACRO{\U{3a9}}%
%BeginExpansion
\Omega%
%EndExpansion
}=\allowbreak 4.\,\allowbreak 32\unit{%
%TCIMACRO{\U{3a9}}%
%BeginExpansion
\Omega%
%EndExpansion
},$ so our resistance is $\left( 432\pm 4\unit{%
%TCIMACRO{\U{3a9}}%
%BeginExpansion
\Omega%
%EndExpansion
}\right) .$
\end{enumerate}

We won't memorize the resistor code, but you should be able to find a
resistance using the code.

If you are in doubt about what color you see on a resistor, our multimeters
can measure resistance directly. 
Place the red probe in the
connector with a $\Omega $ marked on it and turn the dial to the $\Omega $
setting, as seen in Figure \ref{fig:multimeter_ohms}.
Place the probes on either side of the resistance to be measured.
Be careful/ You are a resistor too. If you touch your hands to the probes
(common mistake while you try to hold the resistor on the probe ends) you
may measure your resistance instead of the resistor's! You have a resistance
of around half a megaohm. This is a general concern, every time you measure
resistance with a meter you need to take the circuit element (resistor,
light bulb, whatever) out of the circuit and measure it on it's own.
Otherwise, you might be measuring the resistance of the rest of the circuit.
Alligator clips are useful for this.
\begin{figure}[htbp!]
\centering
\includegraphics[width=\textwidth]{PH4CAU13}
\caption[Multimeters prepared to measure resistance]{Multimeters prepared
to measure resistance.}
\label{fig:multimeter_ohms}
\end{figure}

\subsubsection{A Note on the Direction of Current Flow}


% FIXME HERE
There is a historical oddity with current flow. That is that the current
direction is the direction positive charges would flow. This may seem
strange, since in good conductors electrons are doing the flowing and they
are negative! The electrons go the opposite way the current goes. \begin{figure}[h!]
\includegraphics[width=3.3373in,height=1.5696in]{PH4CAU14}
\end{figure}%
The truth is that it is very hard to tell the difference between positive
charge flow and negative charge flow the other direction. In fact, only one
experiment that I know of shows that the charge carriers in metals are
electrons. And mathematically, the flow of electrons one direction is
equivalent to the flow of positive charges the other direction.\begin{figure}[h!]
\includegraphics[width=%
3.8994in,height=3.3961in]{PH4CAU15}
\end{figure}%
Worse yet, in biological things it \emph{is} positive ions that flow. So for
biology a positive charge carrier is just fine.

Ben Franklin chose the direction we now use. He had a 50\% chance of making
it easy for our electronics lab. But he got it backwards for us (but right
for biology--and how many electronic things did Ben Franklin have anyway?).
All this shows just how hard it is to deal with all these things we can't
see or touch that we study in PH 220.

And even more importantly, in semiconductors--special electronic devices in
all computers and in our Arduinos--it \emph{is} positive charge that flows.
In many electrochemical reactions \emph{both} positive and negative charges
flow. So Mr. Franklin was not really so very wrong. We will stick with the
convention that \textbf{the current direction is the direction that positive
charges would flow regardless of the actual charge carrier motion. }If you
are like me, this will seem a little backwards, but we all get used to it.

But what makes the electrons or positive charges want to flow in the first
place? We know the answer to this from earlier in this lab reading. It is
potential energy. When we connect a metal wire to the terminals of a battery
we know that the charges in the metal wire ends will experience a difference
in potential energy. The potential energy difference will set up an electric
field inside the conductor. \begin{figure}[h!]
\includegraphics[width=4.2832in,height=1.0253in]{PH4CAU16}
\end{figure}

This field makes the free charges move! It causes a force on the little
electrons. We won't have to measure any fields in our lab today, but you
should know they are there. The important thing is to realize that voltages
produce currents. And the amount of current is proportional to the amount of
voltage. This is just Ohm's law! 
\begin{equation*}
\Delta V=IR
\end{equation*}%
The constant of proportionality is related to how much friction there is for
the charges in the wire.%
\begin{equation*}
I=\frac{1}{R}\Delta V
\end{equation*}%
It is just the resistance, $R$.\ 












Notice that this is important! We have found a way to relate our new
quantity that we want to measure, current, to a voltage. We know how to
measure a voltage! Our Ohm's law equation even tells us what extra part we
need to convert our voltmeter into a current measuring instrument. We will
need a resistor.

\subsection{Knowing the Physics, Design the new instrument}

Now that we understand electrical current, we have some hope of figuring out
how to build an instrument to measure that electrical current. From what we
learned, consider adding in an additional small resistor in our circuit. If
we take a small resistance, one that is small compared to all the other
resistances in the circuit, and we put it in the circuit it will slow down
the current, but not by very much. If the resistance is small enough, we
won't even notice the change. Then if we measure the voltage across that
small resistor with a voltmeter, we could mathematically calculate how much
current we have. Notice that this instrument design has two parts. The first
is adding some new hardware to our voltmeter (a resistor) and the second is
adding in some calculation to get our voltmeter reading converted into
current. 
\begin{equation*}
I=\frac{1}{R}\Delta V
\end{equation*}

Let's give this additional small resistor a name. Let's call it the
\textquotedblleft shunt resistor.\textquotedblright\ 
\begin{equation*}
I=\frac{\Delta V_{meter}}{R_{shunt}}
\end{equation*}%
Today we will have to do the calculation part by hand. In future labs, we
would carefully plan for this calculation in our Arduino sketch code.

\begin{figure}[h!]
\includegraphics[width=4.9156in,height=3.9176in]{PH4CAU17}
\end{figure}

When we are done wiring our new instrument, we will have done something
really cool. We have turned our current measurement into a voltage
measurement. We measured something new in terms of a measurement we already
knew how to make. We will generally try to do this for any type of
measurement. That is because we are very good at measuring voltage, and not
so good at measuring other things electronically.

\subsection{Testing the new instrument}

We will need a way to test how good our new instrument works. And
fortunately we know our multimeters can also measure current. So we can
build our new instrument an compare it to the measurement made by a
multimeter. 
\begin{equation*}
\unit{A}=\frac{\unit{V}}{\unit{%
%TCIMACRO{\U{3a9}}%
%BeginExpansion
\Omega%
%EndExpansion
}}
\end{equation*}%
Recall that to use an ammeter (the new instrument we build, or the one in
our multimeter), you must break the electric circuit by disconnecting a
wire. Then you replace that wire with the ammeter. Notice that in the
diagram below that the bottom wire is now broken. \begin{figure}[h!]
\includegraphics[width=1.9372in,height=%
1.471in]{PH4CAU18}
\end{figure}Where a wire was, I have drawn an
ammeter. The current must flow through the ammeter for us to measure it.

\begin{figure}[h!]
\includegraphics[width=4.858in,height=2.2423in]{PH4CAU19}
\end{figure} Remember, to use our multimeters
to measure current, we must turn the dial to the $10\unit{A}$ setting AND
move the red probe to the $10\unit{A}$ connector. Failure to do this may
result in the fuse blowing. Our meter does not warn you that it lost a fuse,
it just pays you back by giving really wrong answers. You should be careful
to connect it right, and be sure it is working (that someone else has not
blown the fuse before you). Since we have different kinds of multimeters, a
second is pictured to the right. For this type of meter, put the red lead in
the connector marked $A$ and turn the dial to the $A$ setting. If the
currents you are measuring are very small, you might have to switch settings
once again. Tiny currents can be measured by moving the dial to the $\unit{mA%
}$ setting AND changing the red probe to the $\unit{mA}$ connector.\begin{figure}[h!]
\includegraphics[width=4.9769in,height=2.2928in]{PH4CAU1A}
\end{figure}

Our multimeters really measure current in much the same way we are talking
about for our new Arduino ammeter. They have a series of shunt resistors
inside of them. When we choose a current measurement setting we are choosing
a shunt resistor to put in the circuit (inside the meter, but the meter is
in the circuit). Then the voltmeter will measure the voltage across that
resistor and use that voltage to calculate the current.

If we create the current meter ourselves, we have to know the resistance
that we used! That allows us to use the voltage meter to calculate the
current,%
\begin{equation*}
I=\frac{\Delta V_{meter}}{R_{shunt}}
\end{equation*}%
but our multimeters are programed to know their own shunt resistances and to
do this calculation for us.

If you have time (and some won't) in today's lab, we will build the new
Arduino instrument to measure current , and we will test it with a
multimeter set in ammeter mode. We will put both in the circuit at the same
time (see figure \ref{New Instrument and Test}) so we get readings from
both. Then we can compare and see how well our new instrument works!

\section{Calculating uncertainty, a review}

That is all the new material for today's lab. But I wanted to remind you of
something you already know.

Back in PH150 you should have gotten a good deal of experience in making
measurements. We will be going back to experimentation soon, and we will
need to remember what we learned in PH150 to take the measurements so that
we can interpret our experimental results. You will remember that every
measurement has an uncertainty. We have to estimate that uncertainty. It
turns out that our voltage measurement schemes will introduce a new source
of uncertainty! And we will have to include this in our uncertainty
calculations. We will take that on next lab, but in this lab let's review
how to calculate uncertainties.

This is a \textquotedblleft review.\textquotedblright\ How much of a
\textquotedblleft review\textquotedblright\ it is may depend on where and
when you took PH150 or its equivalent. If you are a chemist, you will note
that our treatment of uncertainty goes beyond what you learned in
Quantitative Analysis. Let's start by reviewing what a derivative is.

For our purposes, a derivative is a slope of a line. You should recognize
the equation of a straight line as%
\begin{equation*}
y=mx+b
\end{equation*}%
The slope $m$ can be written as 
\begin{equation*}
m=\frac{dy}{dx}
\end{equation*}%
This is nothing magic (or new). It is just a strange way to write $m.$ With
the slope written this way, the equation of the line could be written as 
\begin{equation*}
y=\frac{dy}{dx}x+b
\end{equation*}%
But why $dy/dx$? Think of how we find a slope of a line. Back in junior high
school we called the slope the \textquotedblleft rise over
run.\textquotedblright\ That is, the change in $y$-value divided by the
change in the $x$-value.%
\begin{equation*}
m=\frac{y_{2}-y_{1}}{x_{2}-x_{1}}
\end{equation*}%
In physics, we write the change in a variable using the greek letter delta, $%
\Delta .$ So we could write the slope as%
\begin{equation*}
m=\frac{y_{2}-y_{1}}{x_{2}-x_{1}}=\frac{\Delta y}{\Delta x}
\end{equation*}%
\ Just to jog your memory, let me write out $\Delta y$%
\begin{equation*}
\Delta y=y_{2}-y_{1}
\end{equation*}%
and $\Delta x.$ 
\begin{equation*}
\Delta x=x_{2}-x_{1}
\end{equation*}%
So our straight line equation should be written 
\begin{equation*}
y=\frac{\Delta y}{\Delta x}x+b
\end{equation*}%
but if we take $\Delta x$ to be very, very small it is customary to write
the $\Delta x$ as just $dx$ (I guess a \textquotedblleft $d$%
\textquotedblright\ is smaller than a \textquotedblleft $\Delta $%
\textquotedblright ). If this is not familiar from Math 112, is should be by
now from PH121.

In PH121 you learned that the velocity is the slope of the plot of $x$ vs. $%
t,$ for example, 
\begin{equation*}
y=\frac{1}{2}\frac{\unit{m}}{\unit{s}}t+1\unit{m}
\end{equation*}%
is an equation giving the $y$ position of an object as a function of time.
Note that it is a straight line on a $y$ vs. $t$ plot. \begin{figure}[h!]
\includegraphics[width=%
3.4999in,height=1.868in]{PH4CAU1B}
\end{figure}The slope of the line is 
\begin{equation*}
\frac{dy}{dt}=\frac{1\unit{m}}{2\unit{s}}
\end{equation*}%
We can verify that this works by looking at the plot and noting that for
every two units of time, we go up one position unit. The slope is $1/2\frac{%
\unit{m}}{\unit{s}}.$

But not all curves are straight lines. What do we do with curves that, well,
curve?

One idea is that we could split up the curve into little line segments, each
with its own slope. We can think of $dy/dt$ as an instantaneous slope, a
slope of one of the tiny line segments that make up our curve. This is the
sort of speed measurement that your speedometer gives. The speed might be
different a short time later. But right now the speed is, say, $0.5\unit{m}/%
\unit{s}.$

\begin{figure}[h!]
\includegraphics[width=3.4385in,height=2.3307in]{PH4CAU1C}
\end{figure}Really, in defining an
instantaneous slope we have assumed that the slope near our point on the
curve is essentially a straight line if $\Delta t$ is small enough.

We can use this idea to interpret our error calculations. Suppose I\ throw a
ball in the air with a initial speed of $4\unit{m}/\unit{s}$ straight up
starting from $y_{o}=0$. From PH121 you have learned that the equation for
predicting how high the ball will go is 
\begin{equation*}
y=y_{o}+v_{o}t+\frac{1}{2}at^{2}
\end{equation*}%
It says that starting at $y_{o}$ the ball will go higher depending on the
initial velocity, $v_{o},$ and the acceleration, $a.$ That makes sense.

At a time, $t,$ the ball should be at 
\begin{equation*}
y=0+4\frac{\unit{m}}{\unit{s}}t-\frac{1}{2}\left( 9.8\frac{\unit{m}}{\unit{s}%
^{2}}\right) t^{2}
\end{equation*}%
where $a=-9.8\frac{\unit{m}}{\unit{s}^{2}}$ is the acceleration due to
gravity. So, knowing this, I could predict how high the ball would go if I\
pick a particular time, say, $0.15\unit{s}.$ The result should be%
\begin{eqnarray*}
y &=&0+4\frac{\unit{m}}{\unit{s}}\left( 0.15\unit{s}\right) -\frac{1}{2}%
\left( 9.8\frac{\unit{m}}{\unit{s}^{2}}\right) \left( 0.15\unit{s}\right)
^{2} \\
&=&0.489\,75\unit{m}
\end{eqnarray*}%
This is shown in the next figure with a black line. Solving the equation for 
$y$ is equivalent to drawing a line up to the curve, then from our spot on
the curve over to the $y$-axis to find the position.\begin{figure}[h!]
\includegraphics[width=3.3434in,height=%
2.2295in]{PH4CAU1D}
\end{figure}

For our case, we plot a line upward from $0.15\unit{s}$ to the curve, and
then plot a horizontal line from the intersection to the $y$-axis. We can
see that we get $4.9\unit{m}.$ Suppose I try to verify this by taking a
picture of the ball in flight at $0.015\unit{s},$ but my stop watch is only
good to $\pm 0.005$ seconds. I try to take the picture when the watch is at $%
0.015\unit{s},$ but I might have taken the picture at $0.01\unit{s}$ or at $%
0.02\unit{s}$ or anywhere in between. My time has some uncertainty. What
does the uncertainty in my stop watch time mean for the uncertainty in my $y$
value?

We can get a good approximation by graphically drawing vertical lines up
from $t_{\min }$ and $t_{\max }$ to the curve, and then extending horizontal
lines from the intersections to the $y$-axis. This gives us a $y_{\min }$
and $y_{\max }.$ Our actual height could be anywhere in between these. This
is a way to view our uncertainty in $y.$\begin{figure}[h!]
\includegraphics[width=4.1658in,height=2.7769in]{PH4CAU1E}
\end{figure}

We can use this idea to find a general way to calculate uncertainties. We
could define $\Delta t=t_{\max }-t_{\min }$. If our $\Delta t$ is small
enough (so we can write it just $dt$), the curve is essentially a straight
line in the region between $t_{\min }$ and $t_{\max }.$ So if we knew the
slope of that line (the derivative $dy/dt$) we could easily figure out the $%
y_{\max }$ and $y_{\min }$ points to get our uncertainty range, at least if
we stay near our $t_{n}$ part of the curve. Recall that our uncertainty in $%
y $ is about%
\begin{equation*}
\delta y=\frac{y_{\max }-y_{\min }}{2}=\frac{\Delta y}{2}
\end{equation*}%
Remembering that 
\begin{equation*}
y=\frac{dy}{dt}t+b
\end{equation*}%
then 
\begin{eqnarray*}
\Delta y &=&y_{\max }-y_{\min } \\
&=&\frac{dy}{dt}t_{\max }+b-\frac{dy}{dt}t_{\min }-b \\
&=&\frac{dy}{dt}\Delta t
\end{eqnarray*}%
From PH150, you will recognize this as almost the uncertainty in a function
of one variable! But even if you don't recognize it, we can show that this
is true using our definition of $\delta y$ above. The quantity $\Delta t$ is 
\begin{equation*}
\Delta t=t_{\max }-t_{\min }
\end{equation*}%
so our uncertainty in $t$ would be 
\begin{equation*}
\delta t=\frac{t_{\max }-t_{\min }}{2}=\frac{\Delta t}{2}
\end{equation*}%
then 
\begin{eqnarray*}
\delta y &=&\frac{y_{\max }-y_{\min }}{2} \\
&=&\frac{1}{2}\frac{dy}{dt}\Delta t \\
&=&\frac{dy}{dt}\frac{\Delta t}{2} \\
&=&\frac{dy}{dt}\delta t
\end{eqnarray*}%
so%
\begin{equation*}
\delta y=\frac{dy}{dt}\delta t
\end{equation*}%
So our uncertainty in $y$ is just the slope at our point on the curve
multiplied by our uncertainty in $t.$

But what if we have more than one variable? Say, we have a function $y(x,z),$
we essentially have a two dimensional slope. Think of a hill, you can go
down a hill in more than one direction. So we need slope parts for each
direction we can go.

\begin{figure}[h!]
\includegraphics[width=2.5495in,height=2.0133in]{PH4CAU1F}
\end{figure}%
\begin{equation*}
\Delta y=\frac{dy}{dx}x\hat{\imath}+\frac{dy}{dz}z\hat{k}
\end{equation*}%
But there is a fix we need to make to this equation that you won't learn for
several math classes to come. We want to have a slope in the $x$ and $z$
direction, but we want the slopes to be independent (if you have already
taken PH121, think of two dimensional motion problems, we split the problem
into components). The notation for this is%
\begin{equation*}
\Delta y_{x}=\frac{\partial y}{\partial x}x
\end{equation*}%
\begin{equation*}
\Delta y_{z}=\frac{\partial y}{\partial z}z
\end{equation*}%
where 
\begin{equation*}
\frac{\partial y}{\partial x}
\end{equation*}%
means the component of the slope just in the $x$ direction. We take a
derivative of the function $y,$ but assume only $x$ is a variable (treat $z$
and all $z$ terms with no $x^{\prime }s$ as constants). This lets us
separate the $x$ and $z$ parts. A special, one variable derivative like $%
\partial y/\partial x$ is called a \emph{partial derivative} because you
only take one dimension of the derivative at a time. So, if we wish to find
the error in some general function $z\left( x,y\right) $ the error is given
by 
\begin{equation*}
\delta y=\sqrt{\left( \frac{\partial y}{\partial x}\right) ^{2}\delta
x^{2}+\left( \frac{\partial y}{\partial z}\right) ^{2}\delta z^{2}}
\end{equation*}%
This looks a lot like our slope equation. What we are doing is to assuming
the function $y\left( x,z\right) $ is flat in a small region around the
point we are studying. then the function has a slope $\partial y/\partial x$
in the $x$-direction, and $\partial y/\partial z$ in the $y$-direction. Each
term like 
\begin{equation*}
\left( \frac{\partial y}{\partial x}\right) \delta x
\end{equation*}%
gives how far off we could be in that direction (the $x$-direction in this
case). Remember that we have assumed that $y\left( x,z\right) $ is
essentially flat near our point of interest. The square root may be
something of a mystery, but remember what you have learned about adding
vectors in PH121. We add components of a vector to find the magnitude like
this 
\begin{equation*}
V=\sqrt{V_{x}^{2}+V_{y}^{2}}
\end{equation*}%
This comes from the Pythagorean theorem. The $x$ and $y$ parts of the vector
form two sides of a triangle. We want the remaining side. So we use the
Pythagorean theorem to find the length of the remaining side.

We are doing the same for our small uncertainty lengths. We are just adding
the $x$ and the $y$ components of the error. We could write our error
formula for the general case of a function $f$, that depends on $N$
different variables. 
\begin{equation*}
\delta f=\sqrt{\sum_{i=1}^{N}\left( \frac{\partial f}{\partial x_{i}}\right)
^{2}\delta x_{i}^{2}}
\end{equation*}%
We will use this formula a lot, so make sure you understand what it means
(ask your instructor for help if it is not clear).

\subsection{How do we find the slope?}

But now we have an equation in terms of slope written as $\partial
y/\partial x$ or $\partial y/\partial z$, but how would we ever find these
slopes? Your calculus class has or will teach you how to take a derivative.
They might not have yet taught you how to take this type of derivative. The
symbol $\partial $ means that our derivatives are \textquotedblleft
partial\textquotedblright\ derivatives. This means that we assume all the
variables other than the one that shows up in the derivative symbol are
constants for our derivative.

Let's take an example. What is the slope of the function $y=5zx^{3}?$ if we
calculate the slope only going in the $x$-direction (that is, if we take a
partial derivative with respect to $x$) we get

\begin{equation*}
\frac{\partial }{\partial x}\left( 5zx^{3}\right) =5z\left( 3\right)
x^{3-1}=15zx^{2}
\end{equation*}%
notice that we treated $z$ as a constant! That is what we mean when we user
the symbol $\partial $ and when we say \textquotedblleft partial
derivative.\textquotedblright\ Let's try another. How about finding the
slope of $f=7yx^{2}-2x+z$ with respect to the $x$-direction.

\begin{equation*}
\frac{\partial }{\partial x}\left( 7yx^{2}-2x+z\right) =7y\left( 2\right)
x^{1}-2(1)x^{0}+z\left( 0\right)
\end{equation*}%
The last term illustrates that the slope of a constant is zero, and as we go
just in the $x$-direction, $z$ is constant. That makes sense. So the change
in $f$ just due to the last term $(z)$ should be zero. We also remember $%
x^{0}=1.$ So we are left with 
\begin{equation*}
\frac{\partial }{\partial x}\left( 7yx^{2}-2x+z\right) =14yx-2
\end{equation*}%
We could also find 
\begin{equation*}
\frac{\partial }{\partial y}\left( 7yx^{2}-2x+z\right) =7x^{2}
\end{equation*}%
and%
\begin{equation*}
\frac{\partial }{\partial z}\left( 7yx^{2}-2x+z\right) =1
\end{equation*}

In our equation for calculating uncertainties, we want to find the
uncertainty in each dimension (for each variable) and to add these
uncertainties like components of vectors, so this partial derivative is just
what we want, the slope of our function in just one direction.

\subsubsection{Tie to statistics}

Back in PH150 you should have learned that for experiments where we repeat
the same experiment over and over again, our outcome can be given by the
mean value an our uncertainty can be given by the standard deviation. We
need to tie our statistical ideas into what we have learned about error
propagation. Lets go back to our function $f\left( x,z\right) $ the error is
given by 
\begin{equation*}
\delta f=\sqrt{\left( \frac{\partial f}{\partial x}\right) ^{2}\delta
x^{2}+\left( \frac{\partial f}{\partial z}\right) ^{2}\delta z^{2}}
\end{equation*}%
but now we know we could express this in terms of standard deviations
(provided you don't need to ensure every bit of your data are within your
uncertainty range). We can write our uncertainties as 
\begin{equation*}
\sigma _{f}=\sqrt{\left( \frac{\partial f}{\partial x}\right) ^{2}\sigma
_{x}^{2}+\left( \frac{\partial f}{\partial z}\right) ^{2}\sigma _{z}^{2}}
\end{equation*}%
So one way to get an estimate of uncertainty like $\delta x$ or $\delta z$
above is to make many measurements, and use the standard deviation $\sigma
_{x}$ as an estimate for $\delta x$ and $\sigma _{z}$ for $\delta z.$ This
is usually not too far off (we will refine this analysis in PH336 for those
lucky enough to take the course).

We can use connection between $\delta x$ and $\sigma _{x}$ to show that the
standard deviation of the mean (the best estimate of our uncertainty) is
given by

\begin{equation*}
\sigma _{\bar{x}}=\frac{\sigma _{x}}{\sqrt{N}}
\end{equation*}%
(a result you should have learned back in PH150). Think of calculating a
mean value%
\begin{equation*}
\bar{x}=\frac{x_{1}+x_{2}+\cdots x_{N}}{N}
\end{equation*}%
We can find the uncertainty in this function $\sigma _{\bar{x}}$%
\begin{equation*}
\sigma _{\bar{x}}=\sqrt{\left( \frac{\partial \bar{x}}{\partial x_{1}}%
\right) ^{2}\sigma _{x_{1}}^{2}+\left( \frac{\partial \bar{x}}{\partial x_{2}%
}\right) ^{2}\sigma _{x_{2}}^{2}+\cdots +\left( \frac{\partial \bar{x}}{%
\partial x_{N}}\right) ^{2}\sigma _{x_{N}}^{2}}
\end{equation*}%
You see we just take the partial derivative of our function $\bar{x}$ with
respect to each of the variables $x_{i}$ and multiply by the uncertainty in
that variable written now as a standard deviation $\sigma _{i}.$

For this special case, all of the $x_{i}$ are the same (we are measuring the
same value over and over in taking an average) and all of the $\sigma _{i}$
are the same so we just have%
\begin{equation*}
\sigma _{\bar{x}}=\sqrt{N\left( \frac{\partial \bar{x}}{\partial x_{1}}%
\right) ^{2}\sigma _{x_{1}}^{2}}
\end{equation*}%
and we can take the derivative using our rule. Only $x_{1}$ is a variable,
so we can write the average $\bar{x}$ as 
\begin{equation*}
\bar{x}=\frac{x_{1}}{N}+\frac{x_{2}+\cdots x_{N}}{N}
\end{equation*}%
This is a polynomial! The first term is $\frac{1}{N}x_{1}$ and the whole
second term is a constant if we take a partial derivative with respect to $%
x_{1}$. The derivative is 
\begin{eqnarray*}
\frac{\partial \bar{x}}{\partial x_{1}} &=&\frac{\partial }{\partial x_{1}}%
\left( \frac{x_{1}}{N}+\frac{x_{2}+\cdots x_{N}}{N}\right) \\
&=&\frac{1}{N}x_{1}^{0}+0 \\
&=&\frac{1}{N}
\end{eqnarray*}%
so our statistical error function is just 
\begin{eqnarray*}
\sigma _{\bar{x}} &=&\sqrt{N\left( \frac{1}{N}\right) ^{2}\sigma _{x_{1}}^{2}%
} \\
&=&\sqrt{\frac{\sigma _{x_{1}}^{2}}{N}} \\
&=&\frac{\sigma _{x_{1}}}{\sqrt{N}}
\end{eqnarray*}%
or, since all the $\sigma _{x_{i}}$ are the same, we can just write this as%
\begin{equation*}
\sigma _{\bar{x}}=\frac{\sigma _{x}}{\sqrt{N}}
\end{equation*}

Notice that in this example we had many $x_{i}$ and that to find the
uncertainty we just extended our equation from two variables%
\begin{equation*}
\sigma _{f}=\sqrt{\left( \frac{\partial f}{\partial x}\right) ^{2}\sigma
_{x}^{2}+\left( \frac{\partial f}{\partial z}\right) ^{2}\sigma _{z}^{2}}
\end{equation*}%
to $N$ variables%
\begin{equation*}
\sigma _{f}=\sqrt{\sum_{i=1}^{N}\left( \frac{\partial f}{\partial x_{i}}%
\right) ^{2}\sigma _{i}^{2}}
\end{equation*}

In this special case, we were trying to show a special result, but we can do
this for any function with any number of variables. If your function is
complicated, you just need to take more partial derivative terms under the
square root.

\section{Practice Measurements}

\subsection{Use a Multimeter}

\begin{enumerate}
\item Measure the voltage of a D-Cell battery with a voltmeter. Report the
value you get from the measurement and the uncertainty.

\item Set up the circuit described in section (\ref{Voltage Measurement with
Meter}). The figure is repeated below. Measure the voltage with a
multimeter. \textbf{The indicator on the power supply is not very accurate},
but it can serve as a check to see if we are way off. So compare the power
supply voltage indicator to the meter indicator. Are they the same (you
should think of the uncertainty in both meters and answer using the
uncertainties)
\begin{figure}[h!]
\includegraphics[width=2.9438in,height=1.3353in]{PH4CAU1G}
\end{figure}

\item Modify your circuit described in section (\ref{Voltage Measurement
with Meter}) and change the settings of your multimeter so that you measure
the current in the circuit. Compare your ammeter measurement to the current
shown on the power supply indicator.
\end{enumerate}

\subsection{Use an Oscilloscope}

\begin{enumerate}
\item Predict what you will see on an oscilloscope if you measure the
voltage of a D-Cell battery. Make sure your lab table agrees on your
prediction before you go on.

\item Use the oscilloscope to measure the voltage of a D-Cell battery.

\item Practice interpreting oscilloscope screens

\begin{enumerate}
\item Figure out what the markings on the Oscilloscope screen mean.

\item Figure out what the voltage scale is and how to change it

\item Figure out what the time scale is and how to change it
\end{enumerate}

\item Generate a sinusoidal signal using the stand alone Function Generator.

\item Measure and display this signal using the stand alone digital
oscilloscope.

\item Report the amplitude and the frequency (and their uncertainties) of
the measured signal and compare to the signal generator settings.
\end{enumerate}

\subsection{Build a new instrument from an old instrument}

\begin{enumerate}
\item IF\ THERE\ IS\ TIME, choose a resistor in the $20\unit{k%
%TCIMACRO{\U{3a9}}%
%BeginExpansion
\Omega%
%EndExpansion
}$ range (say, from about $10\unit{k%
%TCIMACRO{\U{3a9}}%
%BeginExpansion
\Omega%
%EndExpansion
}$ to about $30\unit{k%
%TCIMACRO{\U{3a9}}%
%BeginExpansion
\Omega%
%EndExpansion
}$). Choose a shunt resistor in the $200\unit{%
%TCIMACRO{\U{3a9}}%
%BeginExpansion
\Omega%
%EndExpansion
}$ range. Then set up the circuit to measure a current. \begin{figure}[h!]
\includegraphics[width=%
3.8744in,height=3.0874in]{PH4CAU1H}
\end{figure}%
Use an additional multimeter set to measure current to check our
voltage-current measurement. Calculate a percent difference to see how well
this worked.
\end{enumerate}

%TCIMACRO{\TeXButton{\vspace*{\fill}}{\vspace*{\fill}}}%
%BeginExpansion
\vspace*{\fill}%
%EndExpansion
\pagebreak
