%DataLogging
It today's lab, we are going to measure temperature. We will use a transducer that turns temperature (energy of the air molecules around us) into a voltage. So we are still learning to use transducers. But instead of concentrating on validating a physical model for temperature, we are going to concentrate on building an independent system to measure temperature, one that doesn't have to be connected to our computer.

Often we need to have a data collection device that can operate far from our computer, but still save data for later use. For example, if you were to launch your Arduino-instrument on a high altitude balloon.

Today we all need to be able to have our Arduino collect data and save it with no computer attached. We will do this using a shield. An Arduino shield fits over the top of your Arduino, connecting to the necessary pins to do its job. The other pins are still available to use for other measurements. 

\section{Arduino Shields}
	Arduino shields are a simple way to add specific functionality to your Arduino. There are many different shields with many different features. Some shields have additional integrated circuit (IC) chips on them that can add additional computational power.  Others are extremely simple and are designed so you can have a more solid electrical connections with your experiment.  Here is an example of a simple prototyping shield.
	\begin{figure}[h!] 
		\caption{Top of a prototyping shield}
		\includegraphics[width=3.614in,height=2.3981in]{20200309_133406}
	\end{figure}
	\begin{figure}[h!] 
		\caption{Bottom of a prototyping shield}
		\includegraphics[width=3.614in,height=2.3981in]{20200309_133450}
	\end{figure}
	
	Notice the long wire pins on the bottom. These are designed to be plugged in directly into the Arduino. When not in use these should be pushed into the conductive foam. This will not only protect them from getting bent, but will also protect the electrical circuits on the shield from static shocks.
	
	Many shields are compatible with other shields such that they can be stacked onto each other. This will depend on what pins each shield is using and on the physical placement of both the bottom and top pins.
	
	Because shields are designed for a specific functionality the manufactures will often provide example code that utilizes those features. This code can then be modified and/or combined with other code to get your Arduino to do what ever you want. While working with shields be aware of the pins that the shield is using. So you don't interferer with its operation use other pins for your experiments.

\section{Data Logger Shield}
	Data logging is one of the most important parts of any experiment. Data for many experiments often depend on time. The data logger shield allows us to not only save the data to an SD card but we can also save the time along with it. This is because it has a built in real time clock(RTC). 
	
	Take a close look at the figure \ref{Data_Logger}. You will notice that the SD card slot (right next to the words "Data logger Shield") is close to an IC. That IC (a black rectangle) controls the SD card. 
	
	There is also a battery holder. It has a battery in it, but the image also shows a piece of blue paper covering both sides of the battery. This is for shipping so that the battery isn't drained while not in use. The battery is needed for the real time clock. So that it keeps time even if there is no other power to the board. The electronics for the RTC are right next to the battery. Note the smaller IC and the silver cylinder. The cylinder holds a small quartz crystal tunning fork that keeps oscillating. These oscillations are counted by the IC and are used to keep time. 
	\begin{figure}[h!] 
		\caption{Top of Data Logger Shield}
		\label{Data_Logger}
		\includegraphics[width=3.614in,height=2.3981in]{20200310_110036}
	\end{figure}
	
	Just like on your Arduino the pins are all labeled. But notice that some of the pins are labeled with a white square and a back number. These pins (A4, A5, 9,11,12 and 13) are the pins that the shield uses for its functions. Make sure that your experiment doesn't use these pins. Pins A4 and A5 are used for the RTC and pins 9,11,12 and 13 are used for the SD card. The bottom of shield gives additional labels to these pins that describe their function. 



\section{Set up}
	The Data Logger shield uses a library. To get that library follow these steps in the Arduino software.
	\begin{enumerate}
		\item Under "Tools" go to "Manage Libraries ..."
		\item Search for "RTClib"
		\item Find the library by Adafruit and install it.
	\end{enumerate}

    The first time the software is run on the Arduino with the Data Logger shield the RTC needs to be set.  Pull out the paper that keeps the battery from contacting and put the battery back if it is still there. Then upload the code below. Check the serial monitor. If it says that the "RTC has not been set!" or if the date and time are incorrect then you will need to remove the comment back slashes on the line 
    \begin{lstlisting}[language=Arduino]
    rtc.adjust(DateTime(2014, 1, 21, 3, 0, 0));
    \end{lstlisting}
    . But change the numbers to match the current date and time. Then upload the code again. This will set the clock. This only needs to be done once, as the battery will keep time from then on. So put the comment back slashes back in and upload the code again.

\begin{lstlisting}[language=Arduino]
#include <SPI.h> // Allow SPI interfacing, SD Card
#include <SD.h>

#include "RTClib.h" // Adafruit

// Sketch logs data for 30s, then closes file
// SD card won't write until
// file object is closed or flushed
const unsigned long MAX_TIME = 30000;
unsigned long start_ts;
bool running = true;

// SD PIN on our RobotDyn SD/RTC shields
const int CS_PIN = 9;
File logger;
RTC_DS1307 rtc;

// Used in RTClib's DateTime.toString()
const char FORMAT[] = "YYYY-MM-DD hh:mm:ss";
const size_t FORMAT_LEN = (sizeof(FORMAT) / sizeof(char));

char date_buf[FORMAT_LEN];
char filename[24];

int data; // "data"

void setup() 
{
	// Open serial communications and wait for port to open:
	Serial.begin(9600);
	// wait for serial port to connect. 
	// Needed for native USB port only
	while (!Serial); 
	
	Serial.print("Initializing RTC...");
	if (!rtc.begin()) {
		Serial.println("Couldn't find RTC");
		while (1);
	}
	
	//*****************************************************
	// the ! means not. 
	// So this will exicute if the rtc is not running.
	if (!rtc.isrunning()) { 
		Serial.println("RTC has not been set!");
		// following line sets the RTC to the date & time  
		// this sketch was compiled
		// rtc.adjust(DateTime(F(__DATE__), F(__TIME__)));
		// This line sets the RTC with an explicit 
		// date & time,for example to set
		// January 21, 2014 at 3am you would call:
		// rtc.adjust(DateTime(2014, 1, 21, 3, 0, 0));
	}
	
	Serial.println("done!");
	
	Serial.print("Initializing SD card...");
	
	if (!SD.begin(CS_PIN)) { 
		Serial.println("failed!");
		while (1);
	}
	Serial.println("done!");
	
	// Create folder for logs if it doesn't already exist
	if (!SD.exists("/LOGS/"))
		SD.mkdir("/LOGS/");
	
	// find the first file LOGxxx.TXT that doesn't exist,
	// then create, open and use that file
	for (int logn = 0; logn < 1000; logn++) {
		sprintf(filename, "/LOGS/LOG%03d.TXT", logn);
		if (!SD.exists(filename)) {
			logger = SD.open(filename, FILE_WRITE);
			Serial.print("Opened \'");
			Serial.print(filename);
			Serial.println("\' for logging.");
			break;
		}
	}
	if (!logger) {
		Serial.print("Failed to open file!");
		while (1);
	}
	
	// Seed random func with noise
	randomSeed(analogRead(0)); 
	start_ts = millis();
}

void loop () 
{
	if (running) {
		if (millis() - start_ts < MAX_TIME) {
			DateTime now = rtc.now();
			
			memcpy(date_buf, FORMAT, FORMAT_LEN);
			now.toString(date_buf);
			
			// random int is our "data"
			data = random(-100, 100); 
			
			logger.print(date_buf);
			logger.print(",");
			logger.print(now.unixtime());
			logger.print(",");
			logger.println(data);
			
			// write same data to serial
			Serial.print(date_buf);
			Serial.print(",");
			Serial.print(now.unixtime());
			Serial.print(",");
			Serial.println(data);
			
			delay(750);	    	
		}
		else {
			// Time has elapsed. Write to file and close.
			running = false;
			logger.close();
			Serial.println("Closed file.");
		}
	}
}
\end{lstlisting}


The code above saves random data to the SD card. When you have it working check to see that the data file is on the SD card by putting the SD card in your computer and opening the file.  You will need to modify the code such that it saves the temperature.



%\section{Using an SD Card Reader}
%	
%	Some of us succeeded in incorporating a SD card into our Arduino based instruments. But today we will revisit this. To do this we will need a SD card and to wire up an SD card breakout board.
%	\begin{figure}[h!] 
%	\includegraphics[width=3.614in,height=2.3981in]{PH4CAX47}
%	\end{figure}
%	Recall that there are six pins on our SD card reader board. Those six pins need to be wired to the following Arduino pins:
%	
%	\begin{equation*}
%	\begin{tabular}{ll}
%	SD Card Reader & Arduino \\ 
%	GND & GND \\ 
%	VCC & +5V \\ 
%	MISO & Pin 12 \\ 
%	MOSI & Pin 11 \\ 
%	SCK & Pin 13 \\ 
%	CS & Pin 4%
%	\end{tabular}%
%	\end{equation*}
%	
%	The pin names on the SD card reader board are usually on the back of the card.\begin{figure}[h!]
%		
%	\includegraphics[width=4.5152in,height=2.5365in]{PH4CAX48}
%	\end{figure} \begin{figure}[h!]
%	\includegraphics[width=4.1917in,height=3.1548in%
%	]{PH4CAX49}
%	\end{figure}
%	This figure includes a sensor, in this case a temperature sensing thermistor. Of course, we will need a sketch to tell the Arduino what to do with this new hardware. Here is an example:
%	
%	\bigskip
%	\begin{verbatim}
%	//////////////////////////////////////////////////////////////////////////////////////////
%	//////////////////////////////////////////////////////////////////////////////////////////
%	// DataLogger that gives time and voltage and saves it to a SD card.
%	//   
%	//////////////////////////////////////////////////////////////////////////////////////////
%	// Two input simple voltmeter 
%	// will measure 0 to 5V only!
%	// Voltages outside 0 to 5V will destroy your Arduino!!!
%	//////////////////////////////////////////////////////////////////////////////////////////
%	////Load Libraries//////////////////////////////////////////////////////////
%	  #include <SPI.h>     // Serial Peripheral Interface
%	  #include <SD.h>      // SD card library
%	//DEFINE VARIABLES///////////////////////////////////////////////////////////////////////
%	  const int CS_Pin = 4;       // this sets the pin for the CS connection
%	                             // to the SD card reader
%	// we want to have voltage vs time, so make a place to store a time value
%	   unsigned long time;
%	   int delayTime = 1000;    // time to wait in between data points.
%	 // make some integer variables that identify the analog input pins we will use:
%	   int AI0 = 0;
%	   int AI1 = 1;
%	// you also need a place to put the analog to digital converter values 
%	// from the Arduino
%	   int ADC0 = 0;
%	   int ADC1 = 0;
%	// Remember we will have to convert from Analog to digital converter(ADC) units
%	// to volts. We need our delta_V_min just like we did in lab 3 
%	   float delta_v_min=0.0049;   // volts per A2D unit
%	// We need a place to put the calculated voltage
%	   float voltage = 0.0;
%	 
%	 //SETUP/////////////////////////////////////////////////////////////////////////
%	   void setup() {
%	      // Open serial communications and wait for port to open:
%	      Serial.begin(9600);    // the 9600 tells our Arduino how fast to send data
%	      // Check to see if the serial port is working      
%	      while (!Serial) {
%	          ; // wait for serial port to connect. 
%	            // If it is, just keep going (don't do anything)
%	      }
%	      // Send a message to the Serial Monitor telling us that we are starting
%	      //     to use the SD card.
%	      Serial.print("Initializing SD card...");
%	      // See if the card is present and can be initialized.
%	      //    If there is a problem, tell us using the serial monitor.
%	     if (!SD.begin(CS_Pin)) {
%	         Serial.println("Card failed, or not present");
%	         // it didn't start the SD card write, so don't do anything more:
%	         return;
%	     }
%	     // But if the SD card did initialize, tell us on the serial monitor.
%	     Serial.println("card initialized.");
%	   }
%	 
%	//LOOP/////////////////////////////////////////////////////////////////////////
%	   void loop() {
%	   // We will have to turn our voltage numbers into text to send to our file
%	   // let's make a text variable (called a string) for this
%	      String dataString = ""; 
%	   // Read in the voltages in A2D units form the serial port
%	      ADC0 = analogRead(AI0); 
%	      ADC1 = analogRead(AI1);
%	 
%	   // Convert the voltage across the test resistor to voltage 
%	   //   units using delta_v_min
%	      voltage = (ADC1-ADC0) * delta_v_min;
%	 
%	   // get the time stamp
%	      time = millis();
%	 
%	   // make an output string that has the time and voltage
%	      dataString = dataString + String(time) + ", " + String(voltage);
%	     
%	   // open the file. note that only one file can be open at a time,
%	   // so you have to close this one before opening another.
%	      File dataFile = SD.open("datalog.txt", FILE_WRITE);
%	   // If the file is available, write to it, but then close it 
%	   //   right after. We will only keep the file open while we are 
%	   //   writing to it. This is safer. It helps prevent getting files 
%	   //   corrupted, and let's the Arduino keep adding to the file after 
%	   //   a problem.
%	      if (dataFile) {
%	        // print our data point
%	        dataFile.println(dataString);
%	         // close the file for safety
%	        dataFile.close();
%	         // print to the serial monitor too:
%	        Serial.println(dataString);
%	     }
%	     // if the file isn't open, pop up an error:
%	    else {
%	        Serial.println("error opening datalog.txt");
%	    }
%	    delay(delayTime);
%	   }
%	//////////////////////////////////////////////////////////////////////////////////////////
%	//////////////////////////////////////////////////////////////////////////////////////////   
%	 
%	 
%	\end{verbatim}
%	
%	Make sure you understand each line of this code does. We will modify this to include a special sensor today. So you will need to understand exactly what it does. It is a little fancy in that it tries to check to make sure the file is working properly and warns you if something is wrong. But most of the code is comments. So don't be discouraged by the length.
%	
%	Note that this is a very basic sketch. You should consider improving this sketch (there will be suggestions in the lab assignment below). You may need to improve this sketch for your group project.
%
%\section{Time Stamping}
%
%	Very often, we need to know exactly when a data point was taken. We saw this in our RC circuit lab. Saving data to a SD card could be more useful if we knew the collection time of each data point and could save that along with the data point, itself. We could use a stop watch and write it down, but that defeats our goal of having the computer do the data collection. We want the Arduino system to be able to do this on it's own. To do that, we need to add another hardware piece to our data logger. We need to add a clock. 
%	
%	There is a breakout board that is a stand-alone clock. It has it's own battery that keeps the clock going when the rest of the instrument is turned off. When the Arduino starts up, the data logger code can get the time from the breakout board clock. This will allow the data logger to know the exact time for each measurement and to place a time stamp on that measurement when the data point is recorded. These breakout boards are called Real Time Clocks (RTC) and the ones we have are the Adafruit DS3231 Precision RTC breakout boards (https://www.adafruit.com/product/3013).
%	\begin{figure}[h!]
%	\includegraphics[width=5.1214in,height=1.9294in]{PH4CAX4A}
%	\end{figure}
%
%\section{Connecting the RTC to an Arduino}
%	
%	The real time clock uses the I2c bus on the Arduino. A \textquotedblleft bus\textquotedblright\ is a set of pins that can be used by more than one instrument at a time. Each instrument has a special code or \textquotedblleft address\textquotedblright\ to identify it. In your Arduino code, you use this address to tell the Arduino processor which instrument to get data from.
%	
%	The RTC only needs four wires. The wiring is as follows:
%	
%	\begin{equation*}
%	\begin{tabular}{ll}
%	RTC & Arduino \\ 
%	Vcc & 3.3V \\ 
%	GND & GND \\ 
%	SCL & SCL \\ 
%	SDA & SDA%
%	\end{tabular}%
%	\end{equation*}
%	
%	Since this is an Adafruit product, there is a nice tutorial on wiring up the RTC and a library to download and use to make it run (https://learn.adafruit.com/adafruit-ds3231-precision-rtc-breakout/). The Adafruit example code is shown below. I have commented out the section that would set the clock. Hopefully your clock will already be set, so you won't use that part of the code. If not, ask for help from your Instructor or TA.
%	
%	\bigskip
%	\begin{verbatim}
%	// Date and time functions using a DS3231 RTC connected via I2C and Wire lib
%	#include <Wire.h>
%	#include "RTClib.h"
%	 
%	RTC_DS3231 rtc;
%	 
%	char daysOfTheWeek[7][12] = {"Sunday", "Monday", "Tuesday", 
%	    "Wednesday", "Thursday", "Friday", "Saturday"};
%	 
%	void setup () {
%	  Serial.begin(9600); // set up the serial port
%	  delay(3000); // wait for console opening
%	  // now let's start communication with the real time clock
%	  if (! rtc.begin()) {
%	    Serial.println("Couldn't find RTC");
%	    while (1);
%	  }
%	 
%	/* We will use a Raspberry Pi to set all the clocks ahead of time so we won't
%	   Do this part
%	  if (rtc.lostPower()) {
%	    Serial.println("RTC lost power, lets set the time!");
%	    // following line sets the RTC to the date & time this sketch was compiled
%	    rtc.adjust(DateTime(F(__DATE__), F(__TIME__)));
%	    // This line sets the RTC with an explicit date & time, for example to set
%	    // January 21, 2014 at 3am you would call:
%	    // rtc.adjust(DateTime(2014, 1, 21, 3, 0, 0));
%	    }
%	 */ 
%	  
%	}
%	 
%	void loop () {
%	    // use the RTC to get the time
%	    DateTime now = rtc.now();
%	    
%	    // now print out the time in various ways.
%	    Serial.print(now.year(), DEC);
%	    Serial.print('/');
%	    Serial.print(now.month(), DEC);
%	    Serial.print('/');
%	    Serial.print(now.day(), DEC);
%	    Serial.print(" (");
%	    Serial.print(daysOfTheWeek[now.dayOfTheWeek()]);
%	    Serial.print(") ");
%	    Serial.print(now.hour(), DEC);
%	    Serial.print(':');
%	    Serial.print(now.minute(), DEC);
%	    Serial.print(':');
%	    Serial.print(now.second(), DEC);
%	    Serial.println();
%	    
%	    Serial.print(" since midnight 1/1/1970 = ");
%	    Serial.print(now.unixtime());
%	    Serial.print("s = ");
%	    Serial.print(now.unixtime() / 86400L);
%	    Serial.println("d");
%	    
%	    // We also don't need the next part, but it is interesting to know
%	    //  that you can do time calculations with the RTC library functions.
%	    // calculate a date which is 7 days and 30 seconds into the future
%	    DateTime future (now + TimeSpan(7,12,30,6));
%	    
%	    Serial.print(" now + 7d + 30s: ");
%	    Serial.print(future.year(), DEC);
%	    Serial.print('/');
%	    Serial.print(future.month(), DEC);
%	    Serial.print('/');
%	    Serial.print(future.day(), DEC);
%	    Serial.print(' ');
%	    Serial.print(future.hour(), DEC);
%	    Serial.print(':');
%	    Serial.print(future.minute(), DEC);
%	    Serial.print(':');
%	    Serial.print(future.second(), DEC);
%	    Serial.println();
%	    
%	    Serial.println();
%	    delay(3000);
%	}
%	\end{verbatim}
%	
%	\bigskip
%	
%	You will need to modify today's data logger code to use the RTC. You can do this by reading the example code and figuring out how it prints the time. Then modify this to print to our SD card instead of the serial port. 
%	
%	For our final group designed lab, you will likely want a RTC. Because our real time clocks will have to eventually be resent, it is probably not a good idea to solder the breakout board directly into your instrument.

\section{Lab Assignment}
	
	Work in groups of three to five for this set of problems. We have enough equipment for you to each build your own data logger, but work together and don't go on to another step until each team member has completed the previous step.
	
	\begin{enumerate}
		\item Get the Datalogger sheild up and running and set the correct date and time..
		
		\item Modify the code to take data from a thermistor (included in your kit) and do the math to turn the thermal resistance into a temperature. You will have to look at your Arduino kit manual to know how to write this code. You can start with the example code, but you will have to modify it for the thermistor measurement. Record the temperature on the SD card.
		
		\item Remove the SD card after the data collection is complete, and make sure the data makes sense (compare to a thermometer in the room) and that the SD card writing is working.
		
		\item If there is time, try powering your sensor system on a battery to make sure it can operate independently.
		
		\item If there is time, switch to the digital temperature and humidly sensor. Modify your sketch to read in and output both temperature and humidly values. Again you will have to look at the Arduino kit manual to figure out how to do this. Check your data file to make sure all is working
	\end{enumerate}


