%DataLogging
\chapter{DataLogging}

\objectives
{
\item Explain what an Arduino shield is.
\item Use a datalogging shield to record Arduino data to an SD card.
\item Use an external power source with the Arduino.
}

It today's lab, we are going to measure temperature. We will use a 
transducer that turns temperature (energy of the air molecules around us) 
into a voltage (no surprise here, since that's what all of our electronic
measuring devices must do). 

Our focus is to build an independent system to measure temperature, one that 
doesn't have to be connected to our computer. Up to this point, our Arduino
has been powered via the USB connection to our computers, and the data has
been recorded directly on our computers. 
Often we need to have a data collection device that can operate far from our 
computer, but still save data for later use. For example, if you needed to
send a sensor into the atmosphere via a high altitude balloon, you would not
want (or even be able) to send your laptop up with it.

\section{Arduino Shields}

Today we all need to be able to have our Arduino collect data and save it with 
no computer attached. We will do this using a shield. An Arduino shield fits 
over the top of your Arduino, connecting to the necessary pins to do its job. 
The other pins are still available to use for other measurements. 

Arduino shields are a simple way to add specific functionality to your Arduino. 
There are many different shields with many different features. Some shields 
have additional integrated circuit (IC) chips on them that can add additional 
computational power.  Others are extremely simple and are designed so you can 
have a more solid electrical connections with your experiment.  
Figure \ref{fig:shield} shows an example of a simple prototyping shield 
(basically like our prototyping boards, but with the intention that circuit 
elements will be soldered to the board).
Notice the long wire pins on the bottom. These are designed to be plugged 
directly into the Arduino. (When the shield is not in use, these pins should 
be pushed into the conductive foam block. This will not only protect them from 
getting bent, but will also protect the electrical circuits on the shield from 
static shocks.)
\begin{figure}[htbp!] 
	\centering
	\includegraphics[width=3.614in,height=2.3981in]{20200309_133406}\\
	\includegraphics[width=3.614in,height=2.3981in]{20200309_133450}
	\caption[An Arduino prototyping shield]
	{An arduino prototyping shield. The upper image shows the top of the
	shield, and the lower image shows the bottom of the shield.}
	\label{fig:shield}
\end{figure}
	
Many shields are compatible with other shields such that they can be stacked 
on top of each other. Whether or not two shields are compatible depends on 
which pins each shield is using, and on the physical placement of both the 
bottom and top pins.
	
Because shields are designed for a specific functionality, the manufactures 
will often provide example code that utilizes those features. This code can 
then be modified and/or combined with other code to get your Arduino to do what 
ever you want. While working with shields, you must be aware of which pins the 
shield is using. Otherwise, you might interfere with the operation of the 
shield.

\section{Data Logger Shield}
Data logging is one of the most important parts of any experiment. 
In addition to collecting the information from our sensors, we also often want
to record the time at which that data was collected.
The data logger shield allows us to not only save the data to an SD card, but 
we can also save the time along with it. This is because it has a built in 
real time clock (RTC). 
	
Take a close look at the data logging shield shown in Figure 
\ref{fig:Data_Logger}. You will notice that the SD card slot (right next to 
the words "Data logger Shield") is close to an IC. That IC (a black rectangle) 
controls the SD card. 
	
There is also a battery holder. It has a battery in it, but the image also 
shows a piece of blue paper covering both sides of the battery. This is for 
shipping so that the battery isn't drained while not in use. The battery is 
needed for the real time clock, so that it keeps time even if there is no other 
power to the board. The electronics for the RTC are right next to the battery. 
Note the smaller IC and the silver cylinder. The cylinder holds a small quartz 
crystal tunning fork that keeps oscillating. These oscillations are counted by 
the IC and are used to keep time. 
\begin{figure}[htbp!] 
	\centering
	\includegraphics[width=3.614in,height=2.3981in]{20200310_110036}
	\caption[A data logging shield]{A data logging shield, with the top 
	side shown.}
	\label{fig:Data_Logger}
\end{figure}
	
The pins on the shield are labeled, just like on your Arduino. Notice, though, 
that some of the pins are labeled with a white square and a black number. 
These pins (A4, A5, 9,11,12 and 13) are the pins that the shield uses for its 
functions. Make sure that your experiment doesn't use these pins. Pins A4 and 
A5 are used for the RTC and pins 9,11,12 and 13 are used for the SD card. The 
bottom of shield gives additional labels to these pins that describe their 
function. 

\section{Setting up the data logging shield}
The data logger shield uses a set of pre-written code, known as a
``library''. To get that library follow these steps in the Arduino 
software:
\begin{enumerate}
	\item Under ``Tools'' go to ``Manage Libraries ...''
	\item Search for ``RTClib''
	\item Find the library by Adafruit and install it.
\end{enumerate}

The first time the software is run on the Arduino with the data logger shield, 
the RTC needs to be set.  Pull out the paper that keeps the battery from 
contacting, and put the battery back if it is still there. Then upload the 
code below. Check the serial monitor. If it says that the ``RTC has not been 
set!'', or if the date and time are incorrect, then you will need to remove 
the comment back slashes on the line 
    \begin{lstlisting}[language=Arduino]
    rtc.adjust(DateTime(2014, 1, 21, 3, 0, 0));
    \end{lstlisting}
The arguments to the \code{DateTime} function are the year, month, day of the 
month, hour (24 hour time), minute, and second.
Change the numbers to match the current date and time, then upload the code 
again. This should set the clock. This only needs to be done once, as the 
battery will keep time from then on. So put the comment back slashes back in 
and upload the code again.

%\href{https://dtoliphant.github.io/PH250Manual/Code/DataLog.ino}{Download here}
\lstinputlisting[language=Arduino]{Code/DataLog.ino}

The code above saves random data to the SD card. When you have it working, 
check to see that the data file is on the SD card by putting the SD card in 
your computer and opening the file.  You will need to modify the code such that 
it saves the temperature.

\section{Powering the Arduino}

Up to this point, we have powered our Arduino by means of the USB cable attached
to our computer. Clearly, if our objective is remote data logging, we'll need
a different way to provide power.

You undoubtedly have noticed that the Arduino has a cylindrical port on the
same edge as the USB connection. This can be seen in Figure \ref{fig:power}, 
on the right side of the Arduino. You'll also note that your Arduino kit
included a 9 V battery and a connector, also both shown in this figure. The
external power solution is fairly obvious at this point. (However, be advised
that many of the batteries that come in the Arduino kit will already be 
``dead''. If things aren't working as expected, check your battery voltage
using a multimeter.)
\begin{figure}[htbp!]
	\centering
	\includegraphics[width=0.8\textwidth]{Figures/power.jpg}
	\caption[Powering an Arduino with a 9 V battery]
	{Powering an Arduino with a 9 V battery.}
	\label{fig:power}
\end{figure}


\activity
{
Work in groups of three to five for this set of activities. We have enough 
equipment for you to each build your own data logger, but work together and 
don't go on to another step until each team member has completed the previous 
step.
\begin{enumerate}
\item Get the Datalogger sheild up and running and set the correct date and 
	time.
\item Modify the code to take data from a thermistor (included in your kit) 
and do the math to turn the thermal resistance into a temperature. You will 
have to look at your Arduino kit manual to know how to write this code. (This 
is a great exercise in reading the documentation for circuit elements!) You 
can start with the example code, but you will have to modify it for the 
thermistor measurement. Record the temperature on the SD card.
	
\item Remove the SD card after the data collection is complete, and make sure 
	the data makes sense (compare to a thermometer in the room) and that 
	the SD card writing is working.
		
\item Power your sensor system with a battery to make sure it can operate 
	independent of the computer.
		
%\item If there is time, switch to the digital temperature and humidly sensor. Modify your sketch to read in and output both temperature and humidly values. Again you will have to look at the Arduino kit manual to figure out how to do this. Check your data file to make sure all is working
\end{enumerate}
}

\activity
{
	It is also time to start thinking about your final project. We're now
	going to ask you to read ahead, and look through the chapter on 
	writing a proposal. Brainstorm some project ideas with your group,
	settle on one option (it's probably a good idea to run that option
	by your instructor), and start writing a proposal.
}

