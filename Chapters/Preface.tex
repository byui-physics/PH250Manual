%Preface
\chapter*{Preface}
\addcontentsline{toc}{chapter}{\protect\numberline{}Preface}

Experimental physics is the art of testing physical models to see if they 
accurately describe how nature works. During the last few centuries, myriads 
of measuring devices have been designed and built to make the detailed 
measurements needed to perform this testing. Today, in our digital society,
we would like to interface these measuring devices to computers, so that the 
data can be analyzed and displayed on our computers. In other words, we need 
computer interfacing.

Computer interfacing involves the design and building (or purchase) of sensors,
as well as the hardware necessary to collect data from those sensors and then
send the data to the computer. These tasks normally fall under the purview of
the discipline we call ``engineering'', but they are also necessary skills for
the experimental physicist. These are skills that we wish to tackle in this
laboratory class. 

It is not possible to learn everything there is to know about electronic data
aquisition and analysis in a one semester lab (or even a four semester lab, 
for that matter). Our objective is to learn the basics, providing you with
a solid foundation for further study on your own. In other words, we can make
a good start.

By the time you have completed this course, you will be able to build a simple 
computer interface that collects data from a sensor and saves it to a file on
a computer. You will be developing these skills in the context of introductory
models from the theories of electricity and magnetism.

Designing and building instruments can be difficult sometimes, and is often
frustrating. Things often don't work the way you planned at the beginning. But
building instruments can also be a lot of fun, and one of the best ways to
learn is by digging in and ``getting your hands dirty''. Those difficulties,
errors, and frustrations are often times where you will experience the most
learning and growth.

The authors would like to acknowledge several former students who have
contributed to this text. Tyler Miller spent countless hours testing the 
python and arduino codes included in this text, and also modified the codes
(originally written for use on a PC) to work on Apple computers. Nolan Chandler
wrote the code used to interface with the data logging shield, which includes
setting the real time clock and writing to an SD card.
