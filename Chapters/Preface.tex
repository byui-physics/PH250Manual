%Preface
\chapter*{Preface}
\addcontentsline{toc}{chapter}{\protect\numberline{}Preface}

Experimental physics is the art of testing physical models to see if they 
accurately describe how nature works. During the last few centuries, myriads 
of measuring devices have been designed and built to make the detailed 
measurements needed to perform this testing. Today, in our digital society,
we would like to interface these measuring devices to computers, so that the 
data can be analyzed and displayed on our computers. 

To do this we need to understand computer interfacing.

Computer interfacing involves the design and building (or purchase) of sensors,
as well as the hardware necessary to collect data from those sensors and then
send the data to the computer. These tasks normally fall under the purview of
the discipline we call ``engineering'', but they are also necessary skills for
the experimental physicist. These are skills that we wish to tackle in this
laboratory class. 

It is not possible to learn everything there is to know about electronic data
aquisition and analysis in a one semester lab (or even a four semester lab, 
for that matter). Our objective is to learn the basics, providing you with
a solid foundation for further study on your own. In other words, we can make
a good start.

**** START HERE *****

Students should leave this class with confidence that they can build a simple computer interface that will read in sensor data and save it on a computer.

Students in this class will also spend time investigating physical models from introductory electricity and magnetism theory.

Instrumentation is sometimes difficult, sometimes frustrating, but also a lot of fun. I hope you will find these lab experiences both informative and entertaining. 

I would like to acknowledge Tyler Miller who has spent countless hours testing the codes and writing the Mac versions and helping to improve the text. 
