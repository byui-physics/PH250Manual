%Preface
\chapter*{Preface}
Experimental physics is the art of testing our physical models to see if they really work. Thousands of measuring devices have been designed and built to make the detailed measurements needed to perform this testing. We would like to interface these measuring devices to computers so data collection is all digital. Then the data can be analyzed, and displayed on our computers. To do this we need to understand computer interfacing.

Computer interfacing is a bit of engineering. We need to either design and build, or purchase instruments, and then get data from those instruments into a computer. But even though it is an engineering task, it is a necessary skill for experimental physicists. It is this skill that we wish to take on in this laboratory class. Of course we can't learn all that there is to know on computer interfacing in one semester. But we can make a good start.

We will use the open source Arduino microcontroler board as our computer interface and we will use Python and the Arduino C++ languages to write controlling software.

Students should leave this class with confidence that they can build a simple computer interface that will read in sensor data and save it on a computer.

Students in this class will also spend time investigating physical models from introductory electricity and magnetism theory.

Instrumentation is sometimes difficult, sometimes frustrating, but also a lot of fun. I hope you will find these lab experiences both informative and entertaining. 

I would like to acknowledge Tyler Miller who has spent countless hours testing the codes and writing the Mac versions and helping to improve the text. 
