%Introduction
\chapter*{Introduction}
\addcontentsline{toc}{chapter}{\protect\numberline{}Introduction}

As a PH250 student, you are probably taking PH220 concurrently with PH250. This lab course is designed to teach electronics and computer skills while your PH220 professor teaches electromagnetic field theory. Once you have a little bit of electric field theory under your belt from PH220, then our experiments designed to test out models of electric charge and electromagnetic fields begin in earnest. While we are waiting we will spend some time learning about how to control experiments with a computer, and how to import data from an experiment to a computer.

You should read the material for each lab before the lab begins. There will sometimes be practice problems to do to make sure you will be effective in lab. By preparing before lab you will have the full 2 hours and 45 minutes to make sure you can finish the lab work. Some labs may go fast, but most take the entire lab period. I also suggest you practice your computer and electronics skills a little. Build some blinking lights for your apartment, or measure how loud your roommates are, or something. The Arduino can be the data collection and control part of thousands fun projects.

This class is sometimes frustrating. But it is also a lot of fun. You will be introduced to computer instrumentation and will be able to perform an experiment that you and your lab group design. The student designed experiments are only limited by your imagination and our ability to find equipment. If you have concerns during the semester, don't hesitate to find your instructor or TA and ask.

\section*{Keeping a Lab Notebook}
\addcontentsline{toc}{section}{\protect\numberline{}Keeping a Lab Notebook}

Keeping a detailed and accurate record is a critical laboratory skill. An 
informal version of this record is often referred to as a ``lab notebook'',
because traditionally they were kept in physical paper notebooks. Lab
notebooks would include written notes and observations, equations, and sketches
of the experimental apparatus. With the advent of modern technology, we can
make these lab notebook even better. If kept as an electronic document, you
can easily copy and paste code into the lab notebook, or include photographs
of your experiment.

One requirement of this course is that you submit a lab notebook for each
lab. The criteria on which they will be graded can be found on your course
syllabus or on the learning management system.

\section*{Using This Lab Manual}
\addcontentsline{toc}{section}{\protect\numberline{}Using This Lab Manual}

You should plan on reading the appropriate chapter of the lab manual before 
you come to each lab activity. This will save you a lot of time once you get
into the lab, as you'll already have the ``big picture'' of what we are doing,
and also kind of know how you should proceed. Trying to read along as you go
in lab will simply result in you still being in lab after everyone else has
finished and left.

Each chapter has a defined list of objectives that tell you what you should
be able to do after each lab activity. These objectives are found in a callout
box that appears as follows:
\objectives{
\item Objective no. 1....
\item Objective no. 2....
\item etc....
}

As you begin reading the chapter, pay attention to these objectives. When you
have completed the lab, ask yourself whether you have fulfilled those 
objectives (and whether you would be able to do so again).

Below the objectives, you will find another callout box identifying the 
concepts that you will need to know from your electricity and magnetism 
course. Rather than rehash all of those topics in this text, you should 
simply turn to your E\&M textbook (or another resource) and review any
items that are unfamiliar. The list of review items will appear as follows:
\review{
\item First topic to review....
\item Second topic to review....
\item etc....
}

As you proceed through the chapter, you will encounter places where you 
would now complete an activity in the lab. To make these activities easy to
find, they also can be found in callout boxes that look like the following:
\activity{
This is an example of an activity callout box. It will tell you what you need
to do at this point in the lab.}

% FIXME
