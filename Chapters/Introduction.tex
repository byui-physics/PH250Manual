%Introduction
\chapter{Introduction}
\addcontentsline{toc}{chapter}{\protect\numberline{}Introduction}

How will you get the most value out of this laboratory experience? First of all,
it is important to keep the big picture in mind. The focus of this lab is 
learning some basic electronics and computer interfacing. Finer grained
objectives are provided at the beginning of each chapter, and you will want to
ask yourself throughout each lab whether you are fulfilling those objectives.

It is critical that you read the associated chapter in this text before coming
to each lab. The fact that you were able to enroll in this lab suggests that 
you have already completed, or are currently enrolled in, and introductory
course in electricity and magnetism. This textbook will not cover the basic
ideas of electricity and magnetism to the same level of depth as you would
expect in a typical introductory textbook, but it will provide a review of
some of the important concepts in electromagnetism (E\&M). More importantly, the
readings found in this textbook will walk you through the activities that you
will be performing in lab, and hopefully warn you of most of the potential 
pitfalls you may encounter. 

If the reviews of electromagnetic theory found in this text leave you feeling
a little uncertain, you are encouraged to revisit the appropriate sections of
your introductory E\&M textbook.

In the readings, you will occasionally come across practice problems. These 
problems are provided to give you an opportunity to assess your readiness for
the lab activity.

Each lab has been allotted a block of time totalling approximately three hours.
Some labs will go faster than others, but expect most to take the entire lab 
period. You are also encouraged to practice the skills that you are learning
outside of lab. For example, you could build some blinking lights for your 
apartment, or measure how loud your roommates are, or maybe even build a system
that activates a buzzer if the noise level gets too high. There are literally
thousands of fun projects you could work on. Many students have
found that this sort of extracurricular practice greatly enriches their 
understanding.

The course culminates with a student designed project. You, along with a small
number of your classmates, will design a build a project that involves digital
measurements, data analysis, and computer control. For many, this project is
the highlight of the semester. It is limited only by your imagination and our
ability to find or purchase equipment.

